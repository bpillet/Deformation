\documentclass[a4paper,10pt,draft,makeidx,twocolumn]{amsart}
\usepackage[thm,couleur,draft]{amsdip}

\setlength{\columnsep}{3em}

\begin{document}
Soit $L \subset Z$ une droite twistorielle de l'espace $Z$ des twisteurs d'une variété hyperkahlérienne $M$ de dimension $2n$. Soit $O \in L$.

\subsection{Le fibré normal}
Le fibré normal de $L$ dans $Z$, noté $N$ s'identifie à $\Oo(1) \otimes \C^{2n}$, ses sections globales forment donc un $\C$-ev de dimension $4n$ qui s'identifie naturellement aux polynômes de degré $1$ à coefficients dans $\C^{2n}$.
\begin{equation}
H^0(L,N) \simeq H^0(\Pro^1, \Oo(1)) \otimes \C^{2n} \simeq \C[\zeta]^1 \otimes \C^{2n}
\end{equation}
Modulo cette identification, une section globale $s$ de cet espace est donc donnée par
\begin{equation}
s(\zeta) = a + \zeta a' \qquad  a,a' \in \C^{2n}
\end{equation}

On construit une base $\beta = (\alpha_i, \alpha'_i)$ de cet espace de la manière suivante~:
\begin{itemize}
\item $\alpha_1, \cdots, \alpha_{2n}$ des sections globales qui évalués en $O$ forment la base canonique de $\C^{2n}$.
C'est-à-dire $\alpha_i(\zeta) = a_i = (\delta_i^j)_j \in \C^{2n}$. 
\item $\alpha'_1, \cdots, \alpha'_{2n}$ des sections globales qui s'annulent en $O$ tandis que leurs dérivées forment la base canonique de $\C^{2n}$.
C'est-à-dire $\alpha'_i(\zeta) = \zeta a'_i = (\zeta \delta_i^j)_j \in \C^{2n}$. 
\end{itemize}

On désignera par $t \in \C^{4n}$ une section de $H^0(L,N)$ vue dans la base $\beta$, au besoin on notera $t = (\tau, \tau') \in \C^{2n} \oplus \C^{2n}$ les composantes sur $\alpha$ et $\alpha'$.

On notera en majuscule les polynômes homogènes en $t$.


Si une fonction $h_i$ est définie sur $W_i$ (resp. $U_i$, $V_i$) on note $h_i(z,w)$ au lieu de $h_i(z_i,w_i)$ (resp. $h_i(z)$ au lieu de $h_i(z_i)$ et $h_i(w)$ au lieu de $h_i(w_i)$)

\subsection{But}
On cherche à construire $\varphi_i(z,t)$ telle que
\begin{enumerate}[(i)]
\item convergence $\Vert \cdot \Vert_\infty$ \label{CVinfty}
\item $[\varphi_i(z,t)]_1 = \sum_s t_s \beta_s(z_i)$\label{Initialisation}
\item Respecte les changements de carte (ou se recolle)\label{Recollement}
\begin{equation}
\varphi_i(g_{ik}(z,\varphi),t) = f_{ik}(z, \varphi)
\end{equation}
\item Condition de domination\label{Domination}
\item Conditions ponctuelle et angulaire\label{Pt-Ang}
\begin{subequations}\begin{gather}
	[\varphi_0(0,t)]^m = [\varphi_0(0,t)]^1  = \sum_{s=1}^{2n} \tau_s \alpha_s(0) = \tau\label{E:Pt}\\
	\left[\dpp{\varphi_0}{z}(0,t)\right]^m = \left[\dpp{\varphi_0}{z}(0,t)\right]^1  = \sum_{s=1}^{2n} \tau'_s \dpp{\alpha'_s}{z}(0) = \tau'\label{E:Ang}
  \end{gather}\end{subequations}
\end{enumerate}

Des fonctions $\varphi_i(z,t)$ satisfaisant \eqref{Initialisation}, \eqref{Recollement},et \eqref{Pt-Ang} sont appelées \emph{solutions formelles}. Elles seront définies comme série formelle en $t$ à coefficients holomorphes en $z$. Sous les hypothèses supplémentaires \eqref{CVinfty}, et \eqref{Domination}, ces séries convergent sur un petit polydisque en $t$ et donnent lieu à une famille de déformations de $L$ dans $Z$.

Pour obtenir la propriété \eqref{Recollement} on essayera de l'obtenir à tous les ordres
\begin{equation}\label{recollement_m}
\left[\varphi_i(g_{ik}(z,\varphi),t)\right]_m = \left[f_{ik}(z, \varphi)\right]_m
\end{equation}
ce qui équivaut à demander
\begin{equation}\label{Recollement_m}
\left[\varphi^m_i(g_{ik}(z,\varphi^m),t)\right]_m = \left[f_{ik}(z, \varphi^m)\right]_m
\end{equation}
Ainsi il semble possible d'obtenir $\varphi$ par récurrence sur $m$.

\subsection{Preuve d'existence de déformation}
On raisonne par récurrence : on définit $\varphi^1$ par l'équation \eqref{Initialisation}, il est clair par définition que $\varphi_1$ satisfait \eqref{Recollement_m} pour $m=1$.

Supposons que l'on ait construit $\varphi^m$ satisfaisant les conditions formelles \eqref{Initialisation}, et \eqref{Recollement_m},et \eqref{Pt-Ang} à l'ordre $m$. Alors on souhaite ajouter à $\varphi^m$ un polynôme homogène $\Phi_{m+1}$ de degré $m+1$ pour que $\varphi^{m+1} = \varphi^m + \Phi_{m+1}$ satisfasse  \eqref{Recollement_m},et \eqref{Pt-Ang} à l'ordre $m+1$.

Par hypothèse de récurrence, la relation de recollement \eqref{Recollement_m}, est satisfaite à l'ordre $m$, ainsi
\begin{equation}
\varphi^m_i(g_{ik}(z,\varphi^m),t) - f_{ik}(z, \varphi^m)
\end{equation}
est une série formelle en $t$ de valuation au moins $m+1$. Soit donc $\Psi_{m+1} (=\Psi_{ik, m+1})$ la partie homogène de degré $m+1$. Kodaira l'appelle \emph{Obstruction d'ordre} $m+1$.

On a la relation suivante (à l'ordre $m+1$)~:
\begin{equation}
\Psi_{ik}(z_k) = \Psi_{ij}(z_j) + \dpp{f_{ij}}{w}(z_j)\Psi_{jk}(z_k)
\end{equation}
que l'on notera abusivement
\begin{equation}
\Psi_{ik} = \Psi_{ij} + \dpp{f_{ij}}{w}\Psi_{jk}
\end{equation}
ou $z \in U_i \cap U_j \cap U_k$ est considéré comme $z_i$, $z_j$ ou $z_k$ suivant le besoin. Et ils sont reliés par les changements de carte de $V$, c'est-à-dire~:
\begin{equation}
z_i = f_{ik}(z_k,0) \quad \dots
\end{equation}

Ainsi $\Psi$ est une section de $H^1(L,N)$ en $z$ à coefficients dans les polynômes homogènes de degré $m+1$ en $t$. Par hypothèse, $H^1 = 0$ dont $\Psi$ est le $1$-cobord  d'une $0$-cochaine $\tilde{\Phi} = (\tilde{\Phi}_{i,m+1}(z_i,t))_i$ à coefficients dans les polynômes homogènes de degré $m+1$ en $t$.
\begin{equation}\label{Cobord}
\Psi_{ik} = \dpp{f_{ik}}{w}\tilde{\Phi}_k - \tilde\Phi_i
\end{equation}
Deux telles cochaînes $\Psi$ diffèrent d'une section globale de $N$ à coefficients dans les polynômes homogènes en $t$ de degré $m+1$ : $H^0(L,N) \otimes \C[t]_{m+1}$.

Que veut-on imposer à $\tilde\Phi_i$ ?
On ne demande qu'à $\tilde\Phi_0$ de satisfaire certaines conditions (car toutes nos hypothèses sont dans la carte "$0$") : les conditions pour que $\varphi^{m+1}$ satisfasse \eqref{E:Pt} et \eqref{E:Ang}.

On peut simplement imposer les mêmes contraintes à $\tilde\Phi_0$ que l'on peut imposer à une section de $H^0(L,N)$ : Fixons un choix de $\tilde\Phi_0$, alors
\begin{itemize}
 \item $\tilde\Phi_0(0,t)$ est un polynôme homogène en $t$ de degré $m+1$ à coefficients dans $\C^{2n}$ vu comme $N_0$ la fibre de $N$ au dessus de $z_0=0$.
\item $\partial_z \tilde\Phi_0(0,t)$ est un un polynôme homogène en $t$ de degré $m+1$ à coefficients dans $\C^{2n}$ vu comme $T_0N_0$ le tangent en $0$ à la fibre de $N$ au dessus de $z_0=0$.
 \end{itemize}
Par les remarques du début, il existe une unique section $U_t$ globale de $N$ sur $L$ à coefficients homogènes de degré $m+1$ en $t$ telle que
\begin{subequations}
\begin{gather}
	U_t(0) = \tilde\Phi_0(0,t)\\
	\partial_z U_t (0) = \partial_z \tilde\Phi_0(0,t)
\end{gather}
\end{subequations}
En prenant dès lors $\Phi := \tilde\Phi - U_t$; Le polynôme homogène de degré $m+1$ en $t$ à coefficient $0$-cochaines $\Phi$ satisfait toujours \eqref{Cobord}.

Ainsi $\varphi^{m+1} = \varphi^m - \Phi$ est un polynôme de degré $m+1$ en $t$ à coefficients dans les fonctions holomorphes en $z$. Il satisfait toujours \eqref{Initialisation}, car il n'a été modifié qu'à l'ordre $m+1 \geq 2$ en $t$.
De plus un calcul rapide permet de voir que
\begin{multline}
\varphi^{m+1}_i(g_{ik}(z,\varphi^{m+1}),t) - f_{ik}(z, \varphi^{m+1}) \\= 
\varphi^{m}_i(g_{ik}(z,\varphi^{m}),t) - f_{ik}(z, \varphi^{m}) - \Phi_{ik,m+1}\\
+o(t^{m+1})
\end{multline}
qui est donc une série formelle en $t$ de valuation au moins $m+2$. Ainsi $\varphi^{m+1}$ satisfait \eqref{Recollement_m} à l'ordre $m+1$.

Reste à vérifier la condition \eqref{Pt-Ang}.

\subsection{Convergence}
\end{document}
