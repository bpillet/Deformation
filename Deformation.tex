\documentclass[a4paper,10pt,draft]{article}
\usepackage[thm,couleur,draft]{dipneuste}

\begin{document}
Soit $L \subset Z$ une droite twistorielle de l'espace $Z$ des twisteurs d'une variété hyperkahlérienne $M$ de dimension $2n$. Soit $O \in L$.

\subsection{Le fibré normal}
Le fibré normal de $L$ dans $Z$, noté $N$ s'identifie à $\Oo(1) \otimes \C^{2n}$, ses sections globales forment donc un $\C$-ev de dimension $4n$ qui s'identifie naturellement aux polynômes de degré $1$ à coefficients dans $\C^{2n}$.
\begin{equation}
H^0(L,N) \simeq H^0(\Pro^1, \Oo(1)) \otimes \C^{2n} \simeq \C[\zeta]_1 \otimes \C^{2n}
\end{equation}
Modulo cette identification, une section globale $s$ de cet espace est donc donnée par
\begin{equation}
s(\zeta) = a + \zeta a' \qquad  a,a' \in \C^{2n}
\end{equation}

On construit une base $\beta = (\alpha_i, \alpha'_i)$ de cet espace de la manière suivante~:
\begin{itemize}
\item $\alpha_1, \cdots, \alpha_{2n}$ des sections globales qui évalués en $O$ forment la base canonique de $\C^{2n}$.
C'est-à-dire $\alpha_i(\zeta) = a_i = (\delta_i^j)_j \in \C^{2n}$. 
\item $\alpha'_1, \cdots, \alpha'_{2n}$ des sections globales qui s'annulent en $O$ tandis que leurs dérivées forment la base canonique de $\C^{2n}$.
C'est-à-dire $\alpha'_i(\zeta) = \zeta a'_i = (\zeta \delta_i^j)_j \in \C^{2n}$. 
\end{itemize}

On désignera par $t \in \C^{4n}$ une section de $H^0(L,N)$ vue dans la base $\beta$, au besoin on notera $t = (\tau, \tau') \in \C^{2n} \oplus \C^{2n}$ les composantes sur $\alpha$ et $\alpha'$.

On notera en majuscule les polynômes homogènes en $t$.


Si une fonction $h_i$ est définie sur $W_i$ (resp. $U_i$, $V_i$) on note $h_i(z,w)$ au lieu de $h_i(z_i,w_i)$ (resp. $h_i(z)$ au lieu de $h_i(z_i)$ et $h_i(w)$ au lieu de $h_i(w_i)$)

\subsection{But}
On cherche à construire $\varphi_i(z,t)$ telle que
\begin{enumerate}[($i$)]
\item convergence $\Vert \cdot \Vert_\infty$ \label{CVinfty}
\item $[\varphi_i(z,t)]_1 = \sum_s t_s \beta_s(z_i)$\label{Initialisation}
\item Respecte les changements de carte (ou se recolle)\label{Recollement}
\begin{equation}
\varphi_i(g_{ik}(z,\varphi),t) = f_{ik}(z, \varphi)
\end{equation}
\item Condition de domination\label{Domination}
\item Conditions ponctuelle et angulaire\label{Pt-Ang}
\begin{equation}
[\varphi_0(0,t)]^m = [\varphi_0(0,t)]^1  = \sum_s t_s \beta_s(0)
\end{equation}
\begin{equation}
\left[\dpp{\varphi_0}{z}(0,t)\right]^m = \left[\dpp{\varphi_0}{z}(0,t)\right]^1  = \sum_s t_s \dpp{\beta_s}{z}(0)
\end{equation}

Des fonctions $\varphi_i(z,t)$ satisfaisant \eqref{Initialisation}, \eqref{Recollement},et \eqref{Pt-Ang} sont appelées solutions formelles. Elles seront définies comme série formelle en $t$ à coefficients holomorphes en $z$. Sous les hypothèses supplémentaires \eqref{CVinfty}, et \eqref{Domination}, ces séries convergent sur un petit polydisque en $t$ et donnent lieu à une famille de déformations de $L$ dans $Z$.

\end{enumerate}
\end{document}